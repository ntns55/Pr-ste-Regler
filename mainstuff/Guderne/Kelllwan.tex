\chapter{Kelllwan}

\begin{table}[H]
    \centering
    \makebox[\linewidth]{           
        \begin{tabular}{|p{0.075\textwidth}|p{0.25\textwidth}|p{0.25\textwidth}|p{0.25\textwidth}|p{0.25\textwidth}|}
            \rowcolor{cerulean!80}\hline
            Niveau & Helbredelse & Gudens retfærdighed & Kelllwans kampånd & Helligt våben\\\hline
        
            1 &  
            2 LP-helbredelse til en person. Kræver berøring& 
            Tag en Troldmands magi fra ham i 10 min.&
            Giver 1 personer +1 Midlertidig LP i 10 minutter&
            1 våben skader med hellig skade. Markeres med et grønt bånd omkring våbnet\\\hline
            
            2 & 
            2 LP helbredelse til 3 forskellige personer. Kræver berøring& 
            Tag magien fra en Troldmand, Magus, Shaman eller Druide i 15 min.& 
            Giver 3 personener +1 Midlertidig LP i 15 minutter& 
            4 våben skader hellig skade. Markeres med et grønt bånd våbnet.\\\hline
            
            3 & 
            Helbred 2 LP til 5 forskellige person. Kræver berøring&
            Tag magien fra en magibruger som ikke er en Kelllwan præst i 20 min.& 
            Giver 3 personer +2 Midlertidig LP i 20 minutter& 
            7 våben skader hellig skade. Markeres med et grønt bånd omkring våbnet Passiv: Kelllwan præsten kan nu bruge rune genstande\\\hline
            
            4 & 
            Fuldt LP til 5 forskellige Personer. Kræver berøring& 
            Tag magien fra en hvilken som helst magibruger i 1 time.& 
            Giver 5 personer +3 Midlertidig LP i 20 minutter&
            Giver Præstens våben evnen, Kelllwans hammer. Denne varer i 5 minutter eller på de første 3 slag. \\\hline
    \end{tabular}
    }
\end{table}

\section{Beskrivelse}
Kelllwan: Retfærdighedens klinge, Lysets messias. Kelllwan er guden for retfærdighed, beskyttelse, selv offer, kamp, loyalitet, ære og ansvar. Blandt hans troende findes mange forskellige folk. Alt fra fattige som søger støtte til loyale krigere som sikrer lande og byer mod orkernes togter og sortelvernes list. Kelllwan sætter også en ære i at udrydde ukyndige magibrugere da dette er efter kejserens bud. Dette gælder for alle magikere der ikke bruger hellig magi.
\subsection{Dogma}
"Beskyt din familie, dine kære, og kæmp for din gud! Brug dine evner til det gode, og beskyt dem, som er svagere end dig. Vær kampklar, og beredt til at nedkæmpe din fjende, selv når du mindst venter det.\\
Udbred mine ord, til selv de mest uciviliserede områder af Kalish. Alle, som udviser loyalitet, fortjener en chance for at blive frelst, beskyttet, og befriet. Alle, som overtræder loven, fortjener at blive jagtet, fanget, og straffet.\\
Kæmp med mod, dø med ære!"

\section{Helbredelse:}
\textbf{Type:} Øjeblikkeligmagi\\ 
\textbf{Kategori:} Berøringsmagi\\
\textbf{Vigtigt:} Folk du helbreder er stadig påvirket af almindelige dødsregler og kan derfor ikke kæmpe i 10 minutter.
\todo{Nedsat healing, men oppet mængde af personer}

\section{Gudens retfærdighed}
\textbf{Type:} Negativmagi\\ 
\textbf{Kategori:} Berøringsmagi\\
Ofret kan ikke kaste magi i den pågældende tid.

\section{Kelllwans kampånd:}
\textbf{Type:} Positivmagi\\ 
\textbf{Kategori:} Berøringsmagi\\
\textbf{Note:} Kelllwan præsten kan ikke selv blive ramt af denne effekt.

\section{Helligt våben}
\textbf{Type:} Positivmagi
\textbf{Kategori:} Berøringsmagi\\
\textbf{Ingredienser:} Grønt bånd\\
Bemærk at denne magi ikke kan kastes på pistoler, pile, buer eller hvilket som helst andet våben der bruger projektiler. Helligt våben kan maksimalt kastes 15 minutter inden kamp, ellers forsvinder effekten. Helligt våben varer indtil personen dør eller indtil personen går ud af kamp. Husk at samle
dine bånd ind efter hver kamp.\\

\subsection{Kelllwans hammer}
Denne magi kan kun bruges på køller eller hamre. Når en person rammes af et våben med der har evnen Kelllwans hammer falder personen bagover. Denne effekt gælder også hvis personen bliver ramt på et skjold. Denne evne kan godt bruges på runevåben.\\ 
\textbf{Kommando:} ”Kelllwans hammer, vælt”
