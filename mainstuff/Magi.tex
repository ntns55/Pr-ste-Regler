\chapter{Magi for Præst}

\section*{Generelle regler om Magi}
\addcontentsline{toc}{section}{Generelle regler om Magi}
Disse regler gælder alle magier på alle tidspunkter hos alle professioner og der findes ingen undtagelser for normale spillere. Hvis der kommer monstre/plotkarakterer i spil, der kan det medføre en overtrædelse reglerne, men dette vil altid blive nævnt ved briefing.\\
\subsection*{Magiens elve bud}
\addcontentsline{toc}{subsection}{Magiens elve bud}
\begin{enumerate}
    \item Alle pegemagier og områdemagier har en maksimal rækkevidde på 5 meter.
    \item Der findes ingen magi der ikke bruger verbale komponenter, dette er ofte i form af en bøn eller en rite.
    \item Riter skal altid siges højt nok til, at dit mål kan høre det. Hører målet ikke riten, har de ret til at ignorere magien.
    \item Magiens kommando skal altid siges højt, så målet kan høre dig.
    \item Alle magier koster det dobbelte af deres niveau i mana at kaste. Dvs. en niveau 1 magi koster 2 mana, imens en niveau 3 magi koster 6 mana.
    \item Antallet af mana en magibruger har, er det samme som hans optjente XP, en magibruger kan maksimalt opnå 18 mana fra hans XP, samt eventuelt tilkøbt mana, skaffet gennem evner eller lignende. 
    \item Når en magibruger går på 0 LP, går han også automatisk på 0 mana.
    \item Hvis en magibruger kaster en magi, kan de kun holde den tilbage i 2 sekunder, før de skal kaste den. Kaster de den ikke, vil den ramme dem selv uanset effekt eller hvilke barrierer, der måtte beskytte dem fra magi.
    \item Alle magier kan bruges i og udenfor kamp. Vær opmærksom på, at når du forklarer en effekt til et offer, at i begge stadig in-game og derved kan tage skade osv. Hvis du er udenfor kamp eller skal kaste en magi på en person, som ligger ned, er det tilladt at bøje sig over personen og sige effekten lavt, således at du ikke forstyrre spillet.
    \item Alle magier går direkte igennem skjold, rustning og våben.
    \item Bruger du mere end 20 mana bør du tage kontakt til en arrangør.
\end{enumerate}

Der findes grundlæggende fire typer af magier:
\begin{itemize}
    \item Negativ magi
    \item Positiv magi
    \item Øjeblikkelig magi
    \item Passiv magi
\end{itemize}

{\large\textbf{Negativ magi}}\\
Magien påfører en negativ effekt på offeret, som varer i længere tid. En spiller kan kun være påvirket af en negativ magi af gangen. Skulle en spiller der allerede er påvirket af en negativ magi blive ramt af endnu en negativ magi, har den sidste magi ingen effekt.

{\large\textbf{Positiv magi}}\\
Magien påvfører en positiv effekt på en spiller, dette kan være i form af mere LP i længere tid eller et skjold. Skulle en spiller der allerede er påvirket af en positiv magi blive ramt af endnu en positiv magi, har den sidste magi ingen effekt.

{\large\textbf{Øjeblikkelig magi}}\\
Dette er magier der kun vare et øjeblik. Dette kan fx være helbredende magier eller magier der giver skade. Du kan være påvirket af uendelig mange øjeblikkelige magier, da disse vare i under et sekund.

{\large\textbf{Passiv magi}}\\
Magiske effekter som ikke kan fjernes. De vil altid være i effekt og koster ikke mana at bruge. De er også gældende hvis du er død.

Derudover findes der fire kategorier af magier:\\
{\large\textbf{Berøringsmagi}}\\
Når du kaster denne type magi skal du berører dit subjekt.

{\large\textbf{Pegemagi}}\\
Når du kaster denne type magi skal du pege på dit subjekt, der ikke må være længere væk end 5 meter.

{\large\textbf{Områdemagi}}\\
Denne magi er markeret med gryn (f.eks. havregryn). Denne magi aktiver når et eller flere subjekter krydser grynet.

{\large\textbf{Kastemagi}}\\
Denne magi kræver at du har en lille bold, rispose eller lignende. Denne skal du kaste på den du gerne vil ramme. Hvis du rammer med denne rispose så rammer magien også. Vær opmærksom på at selvom der ikke er nogen begrænsning på hvor langt du kan kaste denne bold, skal subjektet stadig kunne hører hvad din magi gør.

\subsection*{At kaste magi}
\addcontentsline{toc}{subsection}{At kaste magi}
Det vigtigste ved magikastning er at være velforberedt. Nogle magier har krav om visse ingredienser, som magikasteren selv skal medbringe. Hvis der er tale om en magi, hvor du skal ramme offeret med en genstand, f.eks. en skumbold, skal du sige effekten når genstanden rammer personen. Hvis genstanden rammer et sværd, skjold eller lignende, skal effekten stadig siges, da magien stadig har sin fulde effekt. Det er kun muligt at undgå magier, hvis offeret skal rammes, flytter sig. Dette kan f.eks. ikke lade sig gøre ved en pegemagi.


\section{At spille præst}
Begrebet ”guder” har eksisteret næsten lige så længe som Kalish. Da Misrush og Kalleans tanker ramte hinanden, skabte de Armandos, Gudernes Gud. Få årtusinder senere ophøjede Armandos de første guder, og da han kort derefter begrænsede deres muligheder for at forlade deres hjem, opstod der en ny æra for guderne. Æraen hvor gudernes magt afhang af deres troende. Denne æra eksisterer stadig. og derfor har guderne brug for folk som dig, folk som vil kæmpe gudens sag - til den bitre ende. En meget vigtig del for alle præsteselskaber er at skaffe troende og omvender andre guders troende. Dette er en meget vigtig del af gudernes magtbalance, og det siges, at hvis en gud mister alle sine troende, vil den dø for evigt. Til messer er det også vigtigt at få flest muligt af de troende med, både for at de kan få vejledning og blive afklaret. Hver religion kan kun have én helliggrund. Har man et fysisk tempel, vil dette altid være ens helliggrund. Jo mere man gør ud af sit tempel/helliggrund, jo større er chancen for at din gud belønner dig. En helliggrund skal indvies med et ritual hvor mindst 1 arrangør skal være til stede. Ønsker man at flytte sin helliggrund skal man kontakte en arrangør, hvorefter der holdes et nyt ritual, og arrangøren vil tage stilling til konsekvenserne heraf.

\section{Kaste magi som Præst}
Man kaster en magi ved at sige en bøn. Denne bøn skal minimum være 5 ord for hvert niveau den ønskede magi er (Med undtagelse af niveau 1 som starter på 10 ord.). Hvis man siger det samme ord 2 gange tæller det stadig kun som 1 ord. Hvis du helbreder nogen så husk at informere dem om dødsreglerne, som kan findes i det almindelige regelsæt.\\
\begin{table}[H]
    \centering
    \begin{tabular}{c|c}
        Niveau magi & Antal ord i bøn \\\hline
        1 & 10\\
        2 & 15\\
        3 & 20\\
        4 & 25\\
    \end{tabular}
\end{table}

Hvis man bliver afbrudt i ens bøn, skal man starte helt forfra. Magien vil ikke blive brugt og derfor koster forsøget ikke noget mana.\\
\textit{Eksempel: Jeg vil kaste en niveau 2 magi. Derfor siger jeg en bøn på mindst 15 ord, og da jeg siger ”store” to gange skal jeg mindst sige 16 ord. Man må godt sige mere end de nødvendige ord per niveau, dette er kun et minimum.}

\section{Hellig grund}
En præst har et sted som et helligt for hans gud. Dette sted refereres til som præstens hellig grund. Hver gud kan kun have et sted per land, men dette kan dog flyttes af en præst. For at flytte eller skabe en hellig grund skal præsten udføre et ritual hvor han lægger vægt på sin guds dogma og bud. En hellig grund har ingen fast størrelse, men skal dog godkendes af en arrangør når den bliver lavet.

\section{Genvinde mana som præst}
Man kan regenerere ens mana ved at holde en messe eller prædike for en forsamling. For hvert minut en præst messer eller prædiker får præsten 1 mana. Foregår messen på præstens helliggrund eller tempel genvinder han 2 mana per min.

\section{Stierne}
Der findes 4 stier som hver har forskellige magier. Man kan kun kaste magier fra de stier som man selv har, og kun i det niveau som man har valgt stien i. Hver gang man køber et maginiveau som præst, må man vælge et vist antal stier. Det er meget vigtigt at huske at når først man har valgt en sti kan dette aldrig gøres om. Skriv på dit karakterark hvilke stier du har valgt og hvilket niveau de er i, så du ikke glemmer det.\\
Når man får Gudens Magi Niv. 1 får man niveau 1 i alle stier. \\
Når man får Gudens Magi Niv. 2 må man frit vælge 3 stier man får niveau 2 magierne fra.\\ 
Når man får Gudens Magi Niv. 3 må man frit vælge 2 stier, som man har niveau 2 i, som man nu også får niveau 3 magierne fra.\\ 
Til sidst, når man får Gudens Magi Niv. 4, må man vælge 1 sti, som man har niveau 3 i, som man får niveau 4 magien fra.

\chapter{Esseleia}
\begin{table}[H]
    \centering
    \makebox[\linewidth]{           
        \begin{tabular}{|p{0.075\textwidth}|p{0.25\textwidth}|p{0.25\textwidth}|p{0.25\textwidth}|p{0.25\textwidth}|}
            \rowcolor{cerulean!80}\hline
            Niveau & Helbredelse & Indre fred & Monisme & Naturlig harmoni \\\hline
        
            1 & 
            1 LP helbredelse til 3 person. Kræver berøring. & 
            Helbreder gifte og ophæver negative magier. &
            Tillader præsten at tale med andre væsner, i 30 min, selvom denne ikke burde være i stand til det. Dette er en positiv magi. &
            En gryncirkel med radius på 2,5m. Ingen personer kan trække våben inden for denne. Kan kun laves på hellig grunden \\\hline
            
            2 & 
            2 LP helbredelse til 5 personer. Kræver berøring. & 
            Helbreder gifte, ophæver negative magier og fjerner sygdomme. & 
            Gør at 1 væsen ikke kan kaste magier på andre personer, bære våben eller slås i 15 min. Dette er en negativ magi. & 
            En gryncirkel ned radius på 5m. Ingen personer kan slås inden for denne. Kan kun laves på hellig grunden \\\hline
            
            3 & 
            4 LP til 7 personer. Kræver berøring. &
            Helbreder gifte, negative magier og sygdomme. Derudover giver denne magi 2 personer immunitet over for ærefrygt 15 min. & 
            En gryncirkel med en radius op 2,5m. Alle som står i denne cirkel kan forstå hinanden.
             & 
           En gryncirkel med en radius på 5m. Ingen personer kan slås eller kaste magi inden for denne. Kan kun laves på hellig grunden. \\\hline
            
            4 & 
            Fuld LP til 9 Personer. Kræver berøring & 
            Helbreder gifte, negative magier og sygdomme. Derudover giver denne 4 personer immunitet over for ærefrygt og paralyse i 30 min.& 
            Gør at 2 væsner ikke kan kaste magier på andre personer, bære våben eller slås i 30 min. Dette er en negativ magi& 
           En gryncirkel med en radius på 5m. Ingen kan slås, kaste magi, bære våben eller lyve indenfor denne. \\\hline
    \end{tabular}
    }
\end{table}
\section{Beskrivelse}
Esseleia: Naturens beskytter, Fredens center, Dragernes Gudinde.\\ Esseleia er en forholdsvis ny gud, men ikke desto mindre er alle racer velkommende i hendes tro. Derfor er det ikke unormalt at se mange forskellige typer præster, dette blev blandt andet udtrykt i at den første Esseleia præst var blodork. En Esseleia præst bære aldrig våben, men kan bære skjold og rustning. Denne rustning kan ikke være metal, da dette strider mod Esseleias oprindelse som druide. Esselia har for nyligt lavet en pagt med dragerne, som hun nu taler for.

\section{Helbredelse}
\textbf{Type:} Øjeblikkelig magi \\
\textbf{Kategori:} Berøringsmagi\\
\textbf{Vigtigt:} Folk du helbreder er stadig påvirket af almindelige dødregler og kan derfor ikke kæmpe i 10 minutter

\section{Indre fred}
\textbf{Type:} Øjeblikkeligmagi / Positivmagi\\
\textbf{Kategori:} Berøringsmagi\\
\textbf{Krav:} Dette kan kun påvirke folk, som meditere mens præsten prædiker.

\section{Monisme}
\textbf{Type:} Se magien\\
\textbf{Kategori:} Berøringsmagi\\
\textbf{vigtigt:} Husk at forklare effekten grundigt.


\section{Naturlig harmoni}
\textbf{Type:} Negativmagi\\
\textbf{Kategori:} Områdemagi\\
\textbf{Ingredienser:} gryn\\
Alt efter hvilket niveau du har denne magi i kan du lave en gryncirkel på denne størrelse. Når du står i denne cirkel varer den i 20 minutter eller indtil præsten forlader den. Forlader præsten cirklen vil magien blive ophævet.

\chapter{Ewen}
\begin{table}[H]
    \centering
    \makebox[\linewidth]{           
        \begin{tabular}{|p{0.075\textwidth}|p{0.25\textwidth}|p{0.25\textwidth}|p{0.25\textwidth}|p{0.25\textwidth}|}
            \rowcolor{cerulean!80}\hline
            Niveau & Magiens Sfære & Magiens fontæne & Ewens beskyttelse & Magiens Mester \\\hline
        
            1 &  
            Fjern en negativ magi fra 1 væsen.& 
            Giv 2 personer 1 mana og øg deres maksimum mana med +1 i 10 minutter.&
            1 person bliver immun over for den næste negative magi i 10 min.&
            Du må vælge to niveau 1 magier fra Elementalisten, Mentalisten eller Nekromantikeren. Dette kan ikke være passive effekter \\\hline
            
            2 & 
            Fjern en magisk effekt fra 1 væsen eller genstand. Kommando: "Ophæv magi". & 
            Giv 2 personer 2 mana og øg deres maksimum mana med +2 i 20 minutter& 
            3 personer bliver immune over for den næste negative magi i 10 min.& 
            Du må vælge to niveau 2 magier fra Elementalisten, Mentalisten eller Nekromantikeren. Dette kan ikke være passive effekter.\\\hline
            
            3 & 
            Dupliker en magisk effekt fra et væsen til præsten selv.&
            Giv 2 personer 3 mana og øg deres maksimum mana med +3 i 1 time & 
            Præsten bliver immun over for den næste negative eller øjeblikkelige magi i 10 min. Markeres med et gult bånd.& 
            Du må vælge to niveau 1 og en niveau 2 magi fra Elementalisten, Mentalisten eller Nekromantikeren. Dette kan ikke være passive effekter\\\hline
            
            4 & 
            Gør præsten immun overfor 1 magi resten af spilgangen. Denne magi skal skrives på et gult bånd.& 
            Giv 3 personer +4 mana og øg deres maksimum mana med +5 i 2 Timer& 
            2 personer bliver immune over for den næste negative eller øjeblikkelige magi i 30 min. Markeres med gult bånd. & 
            Du må vælge en niveau 3 og to niveau 1 magier fra Elementalisten, Mentalisten eller Nekromantikeren. Dette kan ikke være passive effekter.\\\hline
    \end{tabular}
    }
\end{table}

\section{Beskrivelse}
Ewen: Magiens stormester, broderskabets herrer. Ewen Uwle var, dengang han var af dødelig æt, høj-elver. Han var blevet udvalgt af Armandos til at skulle balance kampen mellem de oprindelige elvere, de korruptere elvere og dværge, og ikke mindst finde en løsning til Ilsher problemet, med tiden er Ewen dog blevet mere en vogter for magiens sfære, og mange magikastere tilbeder ham derfor.

\section{Magiens Sfære}
\textbf{Type:} Øjeblikkelig magi \\
\textbf{Kategori:} Berøringsmagi\\
\textbf{Vigtigt:} Niveau 4 af denne sti bruger dedikeret mana. Det vil sige at når den er kastet fjerner den 8 Maks mana. Dette er mana som ikke kan genvindes før næste spilgang. Denne magi kan kastes flere gange. Hver gang bruger den maks mana. Midlertidig maks mana kan ikke bruges på dedikeret mana.

\section{Magiens fontæne}
\textbf{Type:} Positivmagi\\
\textbf{Kategori:} Berøringsmagi\\
\textbf{Ingredienser:} Vand\\
\textbf{Info:} Du skal velsigne et glas med vand. Personen du vil kaste denne magi på skal drikke dette glas mens du prædiker på din helliggrund.

\section{Ewens beskyttelse}
\textbf{Type:} Positivmagi\\
\textbf{Kategori:} Berøringsmagi\\

\section{Magiens Mester}
Ewen mestre magien og dette kan man se i hvilke magier du kan kaste som Ewen præst. Når vælger denne magi skal du vælge magier fra Elementalisten, Mentalisten eller Nekromantikeren. For at kaste disse magier kan du enten bruge riter som en troldmand eller bede en bøn på 25 ord. Når du har valgt dine magier kan disse ikke ændres. Magierne vil koste mana alt efter deres niveau. Man kan kun tage en magi man har de foregående niveauer til.

\textit{FX. Hubert er lige startet som Ewen præst. Han får derfor Magiens Mester i niveau 1. Han vælger Vindstød fra Elementalisten, og Sandhed fra Mentalisten.}


\chapter*{Ishtar}\addcontentsline{toc}{chapter}{Ishtar}

\begin{table}[H]
    \centering
    \makebox[\linewidth]{           
        \begin{tabular}{|p{0.075\textwidth}|p{0.25\textwidth}|p{0.25\textwidth}|p{0.25\textwidth}|p{0.25\textwidth}|}
            \rowcolor{cerulean!80}\hline
            Niveau & Helbredelse & Ishtars blik & Hvid magi & Hellig Beskyttelse\\\hline
        
            1 &  
            2 LP-helbredelse til en person. Kræver berøring& 
            Offeret mærker stor smerte i 10 sekunder. Berøringsmagi&
            Du må vælge en niveau 1 magi fra en troldmand, Druide eller Shaman. &
            En gryncirkel med en radius på 2,5m. Kun magi kan gå igennem det. Kan kun laves på ens helliggrund.\\\hline
            
            2 & 
            2 LP helbredelse til 2 personer. Kræver berøring& 
            Offeret mærker stor ærefrygt i 15 sekunder. Berøringsmagi& 
            Du må vælge en niveau 2 magi fra en troldmand, Druide eller Shaman.& 
            En gryncirkel med en radius på 5m. Kun magi kan gå igennem det. Kan kun laves på ens helliggrund. \\\hline
            
            3 & 
            2 LP helbredelse til 4 personer. Kræver berøring &
            Du paralysere offeret i 30 sekunder. Pegemagi. & 
            Du må vælge en niveau 3 magi fra en troldmand, Druide eller Shaman. & 
            En gryncirkel med en radius på 5m. Intet kan gå igennem det. Kan kun laves på ens helliggrund. Kan ophæves med ophæv magi. \\\hline
            
            4 & 
            Fuldt LP til 5 Personer. Kræver berøring& 
            Folk knæler hvor præsten går. 5 personer bliver påvirket af ærefrygt i 30 sekunder. Pegemagi.& 
            Du må vælge en niveau 4 magi fra en troldmand, Druide eller Shaman &
            En gryncirkel med en radius på 5m. Intet kan gå igennem det. Kan laves overalt. Kan ophæves med ophæv magi.\\\hline
    \end{tabular}
    }
\end{table}


\section*{Beskrivelse}\addcontentsline{toc}{section}{Beskrivelse}
Ilsherne er ikke guder, men da de er børn af overguden Misrush og guden Daikia, så gør det dem til demi-overguder. Men som alle søskende så bære det rod i samme magt, men er samtidig meget forskellige. En præst er dedikeret til en enkelt ilsher, mens det kun er præsten for det hvide hus, som har lov til at bede til alle.\\ 
Disse regler gælder specifikt for Ishtar præstinden. Ishtar er den største af alle ilsherne. Hun er den mest magtfulde og alle hendes brødre frygter hendes evner. Den hvide ilsher er så magt fuld at det siges at en mandelig sortelver vil dø, hvis han så direkte på hende. Af denne årsag er det forbudt for en mand at kikke på en person fra det hvide hus. En præstinde, fra det Hvide hus, Mister 1 mana for hvert minut hun prædiker for en anden Ilsher end Ishtar.

\section*{Helbredelse}\addcontentsline{toc}{section}{Helbredelse}
\textbf{Type:} Øjeblikkelig magi \\
\textbf{Kategori:} Berøringsmagi\\
\textbf{Vigtigt:} Folk du helbreder er stadig påvirket af almindelige dødregler og kan derfor ikke kæmpe i 10 minutter


\section*{Ishtars blik}\addcontentsline{toc}{section}{Ishtars blik}
\textbf{Type:} Negativmagi\\
\textbf{Kategori:} Afhænger af niveau\\
\textbf{Kommando:} "Smerte, 10 sekunder", ”Ærefrygt - Knæl, 15 sekunder”, "Paralyse, 30 sekunder", "Ærefrygt - Knæl, 30 sekunder".


\section*{Hvid magi}\addcontentsline{toc}{section}{Sort magi}
\textbf{Type:} Afhænger af valg\\
\textbf{Kategori:} Afhænger af valg \\
\textbf{Effekt:} Når du første gang får et niveau i denne sti må du vælge en anden sti fra enten: Troldmand, Druide eller Shaman. Du kan nu kaste magier fra denne sti som om de var præste magier. Du må ignorer evt ritter og bruge bønner med den passende længde i stedet. Når du først har valgt en sti kan denne ikke ændres igen

\section*{Hellig beskyttelse:}\addcontentsline{toc}{section}{Hellig beskyttelse:}
\textbf{Type:} -\\
\textbf{Kategori:} Områdemagi\\
\textbf{Ingredienser:} Gryn\\
\textbf{Effekt:} Alt efter hvilket niveau du har denne magi i kan du lave en gryncirkel på denne størrelse. Når du står i denne cirkel varer den i 10 minutter eller indtil en person forlader cirklen. Alt efter hvilket niveau du har denne magi på, kan forskellige ting blive blokeret af den. Hvis du kun har niveau 1-2 kan alle magier gå igennem, dog blokerer den stadig for alle fysiske angreb som sværd, spyd og pile. Personer kan heller ikke gå ind. Hvis du har niveau 3-4 kan magier heller ikke gå igennem. Dog kan ”ophæv magi” stadig fjerne denne magi.


\chapter{Kelllwan}

\begin{table}[H]
    \centering
    \makebox[\linewidth]{           
        \begin{tabular}{|p{0.075\textwidth}|p{0.25\textwidth}|p{0.25\textwidth}|p{0.25\textwidth}|p{0.25\textwidth}|}
            \rowcolor{cerulean!80}\hline
            Niveau & Helbredelse & Gudens retfærdighed & Kelllwans kampånd & Helligt våben\\\hline
        
            1 &  
            2 LP-helbredelse til en person. Kræver berøring& 
            Tag en Troldmands magi fra ham i 10 min.&
            Giver 1 personer +1 Midlertidig LP i 10 minutter&
            1 våben skader med hellig skade. Markeres med et grønt bånd omkring våbnet\\\hline
            
            2 & 
            2 LP helbredelse til 3 forskellige personer. Kræver berøring& 
            Tag magien fra en Troldmand, Magus, Shaman eller Druide i 15 min.& 
            Giver 3 personener +1 Midlertidig LP i 15 minutter& 
            4 våben skader hellig skade. Markeres med et grønt bånd våbnet.\\\hline
            
            3 & 
            Helbred 2 LP til 5 forskellige person. Kræver berøring&
            Tag magien fra en magibruger som ikke er en Kelllwan præst i 20 min.& 
            Giver 3 personer +2 Midlertidig LP i 20 minutter& 
            7 våben skader hellig skade. Markeres med et grønt bånd omkring våbnet Passiv: Kelllwan præsten kan nu bruge rune genstande\\\hline
            
            4 & 
            Fuldt LP til 5 forskellige Personer. Kræver berøring& 
            Tag magien fra en hvilken som helst magibruger i 1 time.& 
            Giver 5 personer +3 Midlertidig LP i 20 minutter&
            Giver Præstens våben evnen, Kelllwans hammer. Denne varer i 5 minutter eller på de første 3 slag. \\\hline
    \end{tabular}
    }
\end{table}

\section{Beskrivelse}
Kelllwan: Retfærdighedens klinge, Lysets messias. Kelllwan er guden for retfærdighed, beskyttelse, selv offer, kamp, loyalitet, ære og ansvar. Blandt hans troende findes mange forskellige folk. Alt fra fattige som søger støtte til loyale krigere som sikrer lande og byer mod orkernes togter og sortelvernes list. Kelllwan sætter også en ære i at udrydde ukyndige magibrugere da dette er efter kejserens bud. Dette gælder for alle magikere der ikke bruger hellig magi.
\subsection{Dogma}
"Beskyt din familie, dine kære, og kæmp for din gud! Brug dine evner til det gode, og beskyt dem, som er svagere end dig. Vær kampklar, og beredt til at nedkæmpe din fjende, selv når du mindst venter det.\\
Udbred mine ord, til selv de mest uciviliserede områder af Kalish. Alle, som udviser loyalitet, fortjener en chance for at blive frelst, beskyttet, og befriet. Alle, som overtræder loven, fortjener at blive jagtet, fanget, og straffet.\\
Kæmp med mod, dø med ære!"

\section{Helbredelse:}
\textbf{Type:} Øjeblikkeligmagi\\ 
\textbf{Kategori:} Berøringsmagi\\
\textbf{Vigtigt:} Folk du helbreder er stadig påvirket af almindelige dødsregler og kan derfor ikke kæmpe i 10 minutter.

\section{Gudens retfærdighed}
\textbf{Type:} Negativmagi\\ 
\textbf{Kategori:} Berøringsmagi\\
Ofret kan ikke kaste magi i den pågældende tid.

\section{Kelllwans kampånd:}
\textbf{Type:} Positivmagi\\ 
\textbf{Kategori:} Berøringsmagi\\
\textbf{Note:} Kelllwan præsten kan ikke selv blive ramt af denne effekt.

\section{Helligt våben}
\textbf{Type:} Positivmagi
\textbf{Kategori:} Berøringsmagi\\
\textbf{Ingredienser:} Grønt bånd\\
Bemærk at denne magi ikke kan kastes på pistoler, pile, buer eller hvilket som helst andet våben der bruger projektiler. Helligt våben kan maksimalt kastes 15 minutter inden kamp, ellers forsvinder effekten. Helligt våben varer indtil personen dør eller indtil personen går ud af kamp. Husk at samle
dine bånd ind efter hver kamp.\\

\subsection{Kelllwans hammer}
Denne magi kan kun bruges på køller eller hamre. Når en person rammes af et våben med der har evnen Kelllwans hammer falder personen bagover. Denne effekt gælder også hvis personen bliver ramt på et skjold. Denne evne kan godt bruges på runevåben.\\ 
\textbf{Kommando:} ”Kelllwans hammer, vælt”

\chapter{Morwen}

\begin{table}[H]
    \centering
    \makebox[\linewidth]{           
        \begin{tabular}{|p{0.075\textwidth}|p{0.25\textwidth}|p{0.25\textwidth}|p{0.25\textwidth}|p{0.25\textwidth}|}
            \rowcolor{cerulean!80}\hline
            Niveau & Dødens nåde & Tidens Tråd & Skatter og Død & Dødens beskyttelse \\\hline
        
            1 &  
            Helbred 1 bevidstløs person 1 LP.& 
            Paralyser 1 Zombie&
            Et mål kan betale 1 Fjend eller få -10 NK i 2 timer. &
            En cirkel af gryn med en radius på 2,5 meter. Alle inden for denne cirkel kan ikke dø, men vil i stedet altid være på minimum 1 LP\\\hline
            
            2 & 
            Helbred 3 bevidstløse personer 1 LP.& 
            Paralyser 2 zombier, for hver zombie du paralysere genvind 2 Mana.& 
            Et mål kan betale 5 Fjend eller få -10 NK og Halveret deres maks LP i 4 Timer& 
            En cirkel af gryn med en radius på 5 meter. Alle inden for denne cirkel kan ikke dø, men vil i stedet altid være på minimum 1 LP.\\\hline
            
            3 & 
            Helbred 3 bevidstløse personer 2 LP. Disse personer kan huske hvad der er sket før de døde.&
            Paralyser 1 mål & 
            Et mål kan betale 5 Fjend eller få -10 NK, Halveret deres maks LP og Mana i 4 Timer& 
            En cirkel af gryn med radius på 5 meter. Alle inden for denne cirkel kan ikke dø, men vil i stedet altid være på minimum halvdelen af deres max LP.\\\hline
            
            4 & 
            Helbred 4 bevidstløse personer til fuldt LP. Disse personer kan huske hvad der er sket før de døde, samt hvad der er sket omkring dem, mens de var døde. & 
            Giver præstens våben evnen: Dødens gave. & 
            Et mål kan betale 10 Fjend eller få -10 NK, Halveret deres maks LP og Mana. Denne effekt forsvinder først når skatten betales eller spilgangen slutter.&
            En cirkel af gryn med en radius på 5 meter. Alle inden for denne cirkel kan ikke tage skade.\\\hline
    \end{tabular}
    }
\end{table}

\section{Beskrivelse}
Morwen er en sortelverkvinde, der blev ophøjet til gud. Hun blev dræbt som gud og blev dermed til en ærkedæmon. Med hjælp fra Grådighed og Hovmod blev hun genskabt og har taget pladsen som dødens gud, en neutral og retfærdig gud.\\

\section{Dødens nåde}
\textbf{Type:} Berøringsmagi\\
\textbf{Kategori:} Øjeblikkelig\\
Effekt: En person bliver helbredt fra at have været bevidstløs. Vær opmærksom på at denne magi ikke kan bruges på personer som \textbf{\textit{ikke}} er døde. Husk at folk der er helbredt af denne magi stadig lider under svaghed, og kan derfor ikke kæmpe de næste 10 minutter

\section{Tidens Tråd}
\textbf{Type:} Negativmagi\\
\textbf{Kategori:} Pegemagi\\
\textbf{Kommando:} “Paralyse, 30 sekunder”

\subsection{Dødens gave}
\textbf{Type:} Positivmagi\\
\textbf{Kategori:} Kun præsten\\
Du skærer tidens tråd over for en fjende af din gudinde. Denne magi kan kastes på præstens våben. Når en præst rammer en spiller med et slag vil dette dræbe den pågældende spiller, uanset deres mængde af liv.
Denne effekt gælder IKKE hvis personen bliver ramt på et skjold, og vil også blive brugt. Denne effekt kan ikke kastes på Pistoler, pile, buer eller et hvilket som helst andet våben der bruger projektiler. Vær opmærksom på at denne effekt kun er gældende på det næste slag. Når dette slag rammer vil præsten blive påvirket af svaghed\footnote{Svaghed gør at man ikke kan kæmpe i 10 minutter. Dette er IKKE en magisk effekt, og hvis den undgåes vil du i stedet dø.}.\\
\textbf{Kommando:} “Dødens gave - Død.”

\section{Skatter og Død}
\textbf{Type:} Negativmagi\\ 
\textbf{Kategori:} Berøringsmagi\\
Du giver folk valget mellem at betale de skatter din gudinde har sat op, eller betale den ultimative pris. Du kan automatisk ophæve denne magi hvis de betaler dig senere.\\


\section{Dødens beskyttelse}
\textbf{Type:} - \\
\textbf{Kategori:} Områdemagi\\
Så længe en person står inde i den angivne cirkel kan de ikke dø. Hvis præsten træder ud af cirklen forsvinder den. Hvis en person træder ind i cirklen med mindre liv end hvad de burde, så genvinder de ikke liv.


\chapter*{Orlek}\addcontentsline{toc}{chapter}{Orlek}

\begin{table}[H]
    \centering
    \makebox[\linewidth]{           
        \begin{tabular}{|p{0.075\textwidth}|p{0.25\textwidth}|p{0.25\textwidth}|p{0.25\textwidth}|p{0.25\textwidth}|}
            \rowcolor{cerulean!80}\hline
            Niveau & Blodrus & Livets Ritual & Orleks Navn & Orleks Vrede \\\hline
        
            1 &  
            Præsten selv og en blodork går i blodrus får + 2 LP.& 
            Med dette ritual genvinder 4 grønhuder 3 LP. Under ritualet kan du også ofrer en Ikke-Orlek troende og genvinde 2 mana.&
            Kast "Ærefrygt" på 1 person ved berøring&
            Præsten får +4 nævekamp.\\\hline
            
            2 & 
            Præsten selv og en blodork eller ork får + 3 LP og + 2 Nævekamp.& 
            Med dette ritual genvinder 8 grønhuder 3 LP. Under ritualet kan du også ofrer en Ikke-Orlek troende og genvinde 4 mana.& 
            Kast "Ærefrygt" på 2 personer ved berøring.& 
            Præsten får +8 nævekamp\\\hline
            
            3 & 
            Præsten selv og 3 blodorker eller orker får + 4 LP og + 5 Nævekamp&
            Med dette ritual genvinder 4 grønhuder 6 LP. Under ritualet kan du også ofrer en Ikke-Orlek troende og genvinde 6 mana.& 
            Kast "Ærefrygt" på 1 person ved Pegemagi.& 
            Præsten får +18 nævekamp\\\hline
            
            4 & 
            Præsten selv og 5 grønhuder får + 5 LP og + 10 Nævekamp.& 
            Med dette ritual genvinder 8 grønhuder fuldt LP. Under ritualet kan du også ofrer en Ikke-Orlek troende og genvinde 10 mana.& 
            Kast "Ærefrygt" på 2 person ved Pegemagi.&
            Præstens våben får "Orleks næve" på deres næste 3 slag eller i de næste 3 minutter.\\\hline
    \end{tabular}
    }
\end{table}

\section*{Beskrivelse}\addcontentsline{toc}{section}{Beskrivelse}
Som dødelig regerede Orlek riget A’kastin indtil han blev myrdet af gift. Han favoriserede sit folk, grønhuderne, uden lige, hvilket overraskende nok førte til høj velstand i alle omgivende lande. Alle ikke-grønhuder som levede i A’kastin var dog under slavelignende forhold og alle som prøvede at gøre oprør blev dræbt af Orlek i kamp.

\section*{Blodrus}\addcontentsline{toc}{section}{Blodrus}
\textbf{Type:} Positiv magi\\
\textbf{Kategori:} Berøringsmagi\\
Ved at tilbede Orlek gennem et ritual som minimum varer 5 min, får de grønhuder som er under effekten alt efter niveau +X antal LP og nævekamp. Næste gang orkerne kommer i kamp efter ritualet vil de gå bersærk, dog vil de stadig kunne genkende venner hvis de er grønhuder. Man kan ikke kaste magi når man er i blodrus. \textbf{Man kan som maksimum blive påvirket af Blodrus 1 gang pr. time.}
Blodrusen varer fra kampen begynder og 15 min frem eller til man dør. De LP som man får fra
blodrus er de første man mister efter RP, og kan ikke genvindes.\\
\textbf{VIGTIGT:} Denne magi kan IKKE påvirke gobliner. Man kan kun kaste denne magi 1 gang i timen.\\

\section*{Livets Ritual}\addcontentsline{toc}{section}{Livets Ritual}
\textbf{Type:} Øjeblikkeligmagi\\ 
\textbf{Kategori:} Berøringsmagi\\
Igennem dette ritual som minimum kræver 2 minutter prædiken, du genvinder ikke mana fra denne prædiken, kan en præst genoplive grønhuder. Hvis du under denne prædiken ofre en Ikke-Orlek troende vil du genvinde det angivne antal mana.\\
\textbf{Vigtigt:} Folk du helbreder er stadig påvirket af almindelige dødregler og kan derfor ikke kæmpe i 10 minutter.

\section*{Orleks Navn}\addcontentsline{toc}{section}{Orleks Navn}
\textbf{Type:} Negativmagi\\ 
\textbf{Kategori:} Afhænger af Niveau\\
Tvinger en person i knæ.\\
\textbf{Kommando:} "Ærefrygt, Knæl"

\section*{Orleks Vrede}\addcontentsline{toc}{section}{Orleks Vrede}
\textbf{Type:} Positivmagi\\
\textbf{Kategori:} Berøringsmagi - Præsten selv\\
Denne effekt varer i 15 minutter.\\

\subsection*{Orleks næve}\addcontentsline{toc}{subsection}{Orleks næve} 
Med denne evne kan Orlek præsten ophæve en magi, når han rammer noget/nogen med sit våben. Dette gælder på de næste 3 slag, eller i de næste 2 minutter, alt efter hvad der sker først.\\
Denne effekt gælder også hvis personen bliver ramt på et skjold. Denne effekt kan ikke gennem kastes på Pistoler, pile, buer eller et hvilket som helst andet våben der bruger projektiler.\\
\textbf{Kommando:} "Orleks næve, ophæv magi"\\

\chapter{Raffael Moordet}

\begin{table}[H]
    \centering
    \makebox[\linewidth]{           
        \begin{tabular}{|p{0.075\textwidth}|p{0.25\textwidth}|p{0.25\textwidth}|p{0.25\textwidth}|p{0.25\textwidth}|}
            \rowcolor{cerulean!80}\hline
            Niveau & Dæmon blod & Dæmonernes hjælp & Dæmonernes Redskab & Korrupt Beskyttelse \\\hline
        
            1 &  
            1 LP Helbredelse til 2 personer& 
            Villigt sind&
            Præsten kan fornemme om 1 person oprigtigt er Raffael Moordet tilbeder. Dette skal gøres under et ritual.&
            En gryncirkel med en radius på 2,5m. Kun magi kan gå igennem det. Kan kun laves på ens helliggrund.\\\hline
            
            2 & 
            2 LP helbredelse til 2 personer. Disse personer vil også være immun overfor smerte i 30 min. (Dette tæller som en positiv magi.)& 
            Dæmonisk berøring& 
            Præsten kan fornemme om 1 person oprigtigt er Raffael Moordet tilbeder.& 
            En gryncirkel med en radius på 5m. Kun magi kan gå igennem det. Kan kun laves på ens helliggrund.\\\hline
            
            3 & 
            3 LP helbredelse til 4 personer. Disse personer vil også være immun overfor ærefrygt og Smerte i 30 min. (Dette tæller som en positiv magi).& 
            Åbent Sind& 
            Giver præstens våben effekten: "Livs drikker"på de næste 3 slag eller i de næste 3 minutter. Skal markeres med rødt bånd.&
            En gryncirkel med en radius på 5m. Intet kan gå igennem det. Kan kun laves på ens helliggrund. Kan ophæves med ophæv magi.\\\hline
            
            4 & 
            Fuldt LP og RP til 4 personer. Disse personer vil også være immun overfor ærefrygt, tortur og smerte i 30 min. (Dette tæller som en positiv magi)& 
            Bundet sjæl& 
            Giver præstens våben effekten: "Livs drikker" på de næste 10 slag eller i de næste 5 minutter. Skal markeres med rødt bånd&
            En gryncirkel med en radius på 5m. Intet kan gå igennem det. Kan laves overalt. Kan ophæves med ophæv magi.\\\hline
    \end{tabular}
    }
\end{table}

\section{Beskrivelse}
Raffael Moordet: Dæmonernes konge, Kniven i mørket Raffael Moordet er den intelligente ondskabs
gud, gud for mordere, tyve, røvere og andre lyssky aktiviteter. Men han er også dæmonernes Herre og
mester og kontrollere deres sfære, som er ledet af Hadets dæmon

\section{Dæmon blod}
\textbf{Type:} Øjeblikkeligmagi\\
\textbf{Kategori:} Berøringsmagi\\
\textbf{Vigtigt:} Folk du helbreder er stadig påvirket af almindelige dødregler og kan derfor ikke kæmpe i 10 minutter.\\
Denne magi tæller som et ritual. For at kaste denne magi er det ikke nok at sige en bøn med den korrekte mængde af ord, men de der bliver påvirket af denne magi skal drikke præstens blod.

\section{Dæmonernes hjælp}
Denne sti er bundet tæt sammen med dæmonologen, og en af de mange grunde til at Raf ael Moordets tilbedere bliver set som dæmon tilbedere.\\
\subsection{Villigt sind}
\textbf{Passiv effekt}\\
Du dedikere dig til en dæmon. Vælg en dæmon som du får hjælp af og få denne effekt så længe du opfylder kravene for denne.
\begin{itemize}
    \item \textbf{Hadets Dæmon:} Smerte ved berøring en gang i timen uden at bruge mana. - \textbf{krav:} Du skal have en dæmonisk hånd.
    \item \textbf{Grådighedens Dæmon:} Få dobbelt effekt, bonuser og tid, af alle drikke eller andre spiste genstande (Inklusiv dobbelt skade fra gift)
    \item \textbf{Nattens Dæmon:} Genvind 2 mana hver gang du dræber en person - \textbf{krav:} Størknet blod som tårer på kinderne.
    \item \textbf{Jalousiens Dæmon:} Når du torturere en person kan du stjæle 1 mana per minut, hvis dit offer stadig har mana tilbage - \textbf{Krav:} Du skal bære horn.
    \item \textbf{Smertens og Vanviddets Dæmon:} Når du tager skade genvinder du 1 mana - \textbf{Krav:} Du skal have tydelige ar på kroppen.
    \item \textbf{Kaosets Dæmon:} Du kan nu bruge skjold, hvis du er præst må dette skjold være et tårnskjold. - \textbf{Krav:} Du må ikke bære metal (Inklusiv mønter og nibs)
\end{itemize}

\subsection{Dæmonisk berøring}
\textbf{Type:} Negativ magi\\
\textbf{Kategori:} Berøringsmagi\\
En person som bliver berørt bliver tvunget til at knæle.\\
\textbf{Kommando:} “Ærefrygt - 15 sekunder”

\subsection{Åbent Sind}
Passiv effekt\\
Vælg 2 effekter fra magien \textbf{Villigt sind}.

\subsection{Bundet sjæl}
Denne magi kan kun bruges en gang per spilgang, og skal bruges ved spilstart. En anden spiller skal være klædt ud som dæmon. Denne spiller for lov til at spille din dæmon resten af spilgangen. Denne mana kan ikke genvindes når den er brugt.\\
\textbf{Dæmonen har:}\\
18 LP\\ 
15 NK\\
Kan benytte rustning.\\
Skal være udklædt som dæmon.\\
Når dæmonen har fuld LP vil hans RP begynde at regenerer som ved naturlig helbredelse.\\
Dæmonen genvinder 1 LP eller 1 RP hvert minut udenfor kamp og er Immun overfor alle \textbf{ikke skadende} magier dette inkludere healing og andre boost.\\
Dæmonen tager trippel skade fra hellige våben. \\
\textbf{Kommando:} Forklar effekten MEGET grundigt.

\section{Dæmonernes redskab}
\textbf{Type:} Øjeblikkelig magi\\
\textbf{Kategori:} Berøringsmagi\\
Når du rør en person kan du fornemme om denne person er Raffael Moordet troende eller ej. Dette betyder at du må spørge personen off-game om de er Raffael Moordet troende, og de skal svarer oprigtigt.

\subsection{Livs Drikker}
Denne effekt er en positiv magi, som kun påvirker dig. Denne markeres med et rødt bånd omkring dit våben. Denne effekt kan ikke kastes på Pistoler, pile, buer eller et hvilket som helst andet våben der bruger projektiler.\\ 
Når du rammer nogen, således at de tager skade, får du selv 1 LP. Dette LP er et du genvinder, så hvis du allerede har fuld LP, virker denne effekt ikke.\\

\section{Korrupt beskyttelse}
\textbf{Type:} - \\
\textbf{Kategori:} Områdemagi\\
\textbf{Ingredienser:} gryn\\
Alt efter hvilket niveau du har denne magi i kan du lave en gryncirkel på denne størrelse. Når du står i denne cirkel varer den i 10 minutter eller indtil en person forlader cirklen. Alt efter hvilket niveau du har denne magi på, kan forskellige ting blive blokeret af den. Hvis du kun har niveau 1-2 kan alle magier gå igennem, dog blokerer den stadig for alle fysiske angreb som sværd, spyd og pile. Personer kan heller ikke gå ind. Hvis du har niveau 3-4 kan magier heller ikke gå igennem. Dog kan ”ophæv magi” stadig fjerne denne magi. Bemærk at størrelsen af cirklen måles i radius.
