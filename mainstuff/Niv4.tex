\chapter{Niveau 4}
Du er din guds talerør, du er deres sendebud, deres Apostel. Du hjælper på din guds vegne om det er i form af helbredende kræfter og venlige ord, eller om det er et budskab om vold til et punkt hvor jorden er mudret med dine fjenders blod, det ved kun du.\\
\textbf{Ypperstepræst} er den titel der gives til de øverste præster inden for din guds religion. Du vil blive anset som en af de spirituelle ledere af kirken og kun stå til ansvar overfor din gud.\\
\textbf{Krigspræst} er forskellige fra Tempelkrigere da en krigspræst evner og våben er langt mere effektive til at besejre kættere og vantro. Deres gud står bag deres ord og handlinger.

\textbf{Du skal vælge din sti med omhug, da dette ikke kan ændres efter dette er valgt.}\\


\begin{tabular}{|p{0.3525\textwidth}|p{0.1175\textwidth}|p{0.3525\textwidth}|p{0.1175\textwidth}|}
\hline
\rowcolor{cerulean!80}
 \multicolumn{2}{|c|}{ Ypperstepræst } & \multicolumn{2}{|c|}{ Krigspræst }\\
\hline
\rowcolor{cerulean!40}
    Evne navn & Pris i XP & Evne navn & Pris i XP\\ \hline
    Ekstra NK Niv. 2 & 2 & Brug af Ringbrynje/Pladerustning & 0 \\\hline
    
    Kardinal & 1 & Brug af Tårnskjold &2 \\\hline
    
    Gudens Magi Niv. 4 & 2 & Ekstra LP Niv. 3 & 2 \\\hline
    
    Gudens Velsignelse & 3 & Ekstra NK Niv. 2 & 2 \\\hline
    
    Kirkeskat Niv. 4 & 3 & Gudens Våben & 3 \\\hline 
    & & Reparere Rustning Niv. 1 & 2 \\\hline
\end{tabular}

\section{Evne beskrivelse for Ypperstepræst}

\subsection*{Ekstra NK Niv. 2}
\addcontentsline{toc}{subsection}{Ekstra NK Niv. 2}
Du har et ekstra nævekamp.\\

\subsection{Kardinal}
Når du har kontakt til andre dele af din religion vil du blive anset som værende en af de religiøse ledere.

\subsection{Gudens Magi Niv. 4}
Som præst har du adgang til magi afhænging af hvilken gud du tilbeder. Du har nu adgang til niveau 4 magier indenfor din sti. For mere information omkring dine magier se kapitlet "Magi som Præst".

\subsection{Gudens Velsignelse}
Præsten kan vælge en forkæmper, pr spilgang. som får +2 LP og +3NK. Disse er permanente for spilgangen, så længe forkæmperen følger gudens dogma. kan kastes på 1 person.

\subsection{Kirkeskat Niv. 4}
Når du tjekker ind får du 20 fjend og 2 manasten.


\section{Evne beskrivelse for Krigspræst}

\subsection{Brug af Ringbrynje/Pladerustning}
Du kan nu bruge både ringbrynje og pladerustning. 

\subsection{Brug af Tårnskjold}
Du kan nu bruge tårnskjold.

\subsection*{Ekstra LP Niv. 3}
\addcontentsline{toc}{subsection}{Ekstra LP Niv. 3}
Du har et ekstra liv.\\

\subsection*{Ekstra NK Niv. 2}
\addcontentsline{toc}{subsection}{Ekstra NK Niv. 2}
Du har et ekstra nævekamp.\\

\subsection{Gudens Våben}
Du bruger din guds magt til at overvinde dine fjender. Hvis dit våben er helligt kan du opsuge denne magi ved at bruge 4 mana for at give 5 hellig skade til en fjende. Dette kræver også en bøn på 10 ord. Dit våben er ikke længere helligt, hvis dit våben er helligt fra rune våben vil dette ødelægge runevåbnet. Når du rammer personen siger du følgende kommando: 5 Hellig skade.

\subsection{Reparere rustning Niv. 1}
Du kan genoprette 1 RP på ødelagt rustning, når du benytter 2 min på at reparere denne. Dette kan gentages indtil rustningen fremstår 'uden skader'.\\